%%%%%%%%%%%%%%%%%%%%%%%%%%%%%%%%%%%%%%%%%
% Short Sectioned Assignment
% LaTeX Template
% Version 1.0 (5/5/12)
%
% This template has been downloaded from:
% http://www.LaTeXTemplates.com
%
% Original author:
% Frits Wenneker (http://www.howtotex.com)
%
% License:
% CC BY-NC-SA 3.0 (http://creativecommons.org/licenses/by-nc-sa/3.0/)
%
%%%%%%%%%%%%%%%%%%%%%%%%%%%%%%%%%%%%%%%%%

%----------------------------------------------------------------------------------------
%	PACKAGES AND OTHER DOCUMENT CONFIGURATIONS
%----------------------------------------------------------------------------------------

\documentclass[paper=letter, fontsize=12pt]{scrartcl} % Letter paper and 12pt font size
\usepackage[top=1in, bottom=1in, left=1in, right=1in]{geometry}

\usepackage[T1]{fontenc} % Use 8-bit encoding that has 256 glyphs
\usepackage{fourier} % Use the Adobe Utopia font for the document - comment this line to return to the LaTeX default
\usepackage[english]{babel} % English language/hyphenation
\usepackage{amsmath,amsfonts,amsthm, amssymb} % Math packages
\usepackage{enumitem} % for a) b) ... list
\usepackage{tipa}
\usepackage{xcolor}
\usepackage{lipsum} % Used for inserting dummy 'Lorem ipsum' text into the template
\usepackage{graphicx}
\usepackage{listings}
\usepackage{multicol}
\PassOptionsToPackage{hyphens}{url}\usepackage{hyperref}

\usepackage{sectsty} % Allows customizing section commands
\allsectionsfont{\centering \normalfont\scshape} % Make all sections centered, the default font and small caps

\usepackage{fancyhdr} % Custom headers and footers
\pagestyle{fancyplain} % Makes all pages in the document conform to the custom headers and footers
\fancyhead{} % No page header - if you want one, create it in the same way as the footers below
\fancyfoot[L]{} % Empty left footer
\fancyfoot[C]{} % Empty center footer
\fancyfoot[R]{\thepage} % Page numbering for right footer
\renewcommand{\headrulewidth}{0pt} % Remove header underlines
\renewcommand{\footrulewidth}{0pt} % Remove footer underlines
\setlength{\headheight}{13.6pt} % Customize the height of the header

\definecolor{code-red}{rgb}{0.8,0.25,0.33}
\definecolor{light-gray}{gray}{0.95}
\definecolor{text-gray}{gray}{0.2}
\colorlet{light-pink}{red!5}

\newcommand{\code}[1]{\colorbox{light-gray}{\textcolor{text-gray}{\texttt{#1}}}}
\newcommand{\ih}[1]{\textcolor{purple}{#1}}

\numberwithin{equation}{section} % Number equations within sections (i.e. 1.1, 1.2, 2.1, 2.2 instead of 1, 2, 3, 4)
\numberwithin{figure}{section} % Number figures within sections (i.e. 1.1, 1.2, 2.1, 2.2 instead of 1, 2, 3, 4)
\numberwithin{table}{section} % Number tables within sections (i.e. 1.1, 1.2, 2.1, 2.2 instead of 1, 2, 3, 4)

\setlength\parindent{0pt} % Removes all indentation from paragraphs - comment this line for an assignment with lots of text

%----------------------------------------------------------------------------------------
%	TITLE SECTION
%----------------------------------------------------------------------------------------

\newcommand{\horrule}[1]{\rule{\linewidth}{#1}} % Create horizontal rule command with 1 argument of height

\title{	
\normalfont \normalsize 
\textsc{COMP 533: Object Oriented Software Development} \\ [25pt] % Your university, school and/or department name(s)
\horrule{0.5pt} \\[0.4cm] % Thin top horizontal rule
\huge Concern-Oriented Reuse Project - Phase II\\ % The assignment title
\horrule{2pt} \\[0.5cm] % Thick bottom horizontal rule
}

\author{Jeremie Poisson - 260627104 \\ Anna Jolly - 260447198} % Your name

\date{\normalsize December 15, 2017} % Custom date
%\date{\normalsize\today} % Today's date

\begin{document}

\maketitle % Print the title

\section{Recap of Concern and Features}

We have chosen to develop a new concern entitled \textit{Data Collection}. This concern consists of a joint monitoring/logging system that can be used to collect and log data as an application is used. The monitoring feature provides a way to monitor performance-related and/or user-related events. The \textit{Performance analysis} feature provides time events, which are timestamps of certain actions (e.g. the time a page is requested) and performance events, which include throughput, utilization, and count events. These events can then be aggregated into entries (e.g. the time event of a page request and the time event of when the page is done rendering might be aggregated into an entry). On the other hand, the \textit{User analysis} feature provides a way of monitoring user-initiated events. These events can be page access events, login events, or exit events (closing the app or logging out). The logging feature provides a way for application developers to trace different kinds of events in the system. Logging entries are created from logging events launched by applications developers at some point in their system. Two kinds of logging events can be launched: \textit{TextLoggableEvent} and \textit{ExceptionLoggableEvent}, used to log simple text or an exception, respectively. All events are then used to generate entries, which are the textual representations of such events to be collected in the logs. The \textit{collection} feature enables application developers to specify collectors defining how the generated entries must be collected (in a file, over the network, over an output stream, etc.). The \textit{error reporting} feature enables an end-user of the application to report errors whenever a fatal error causes the application to crash during use. This feature captures the user personal information as well as their system configuration and an optional message.

\section{Flow of Information}

\subsection{Event Generation}

Events can originate from three places, logging (e.g. an exception was generated), user analysis (e.g. a session was started with a login action), or performance analysis (e.g. a resource was used). In the case of logging, the \textit{Logger} is in charge of generating the event automatically. For user analysis, the user can generate their own events in their code, or they can map their objects/functions to the pre-defined aspects of \textit{User Analysis}. In this latter case, the user does not need to deal with events, since they will be generated "behind the scenes". One example of this is the \textit{Login} aspect. The user of our concern needs only to map their login function to that of our \textit{Login} aspect, and at the end of every login, the concern will automatically generate a login event and process it correctly. \\
Events can be regular (single) events, or they can be of type EventSequence. This means that several events are joined together as one event. An example of this is \textit{SessionEventSequence}, which is is an event consisting of a \textit{LoginEvent} and an \textit{ExitEvent}. Again, the user can implement their own type of event sequence that they wish to monitor, or they can map to the pre-existing sequences available: session (as described above), or page access (which is a sequence of accesses to a single page made by a user).

\subsection{Entry Generation}

Once an event is generated, it is passed to the \textit{DataCollectionManager}, which holds several kinds of \textit{EventHandlers}. The manager gives the event to the appropriate event handler, which is charged with generating an entry from the event. This entry is then passed on to the registered \textit{EntryCollector}, which adds the entry to the collector. The \textit{EntryCollector} can be a single collector (e.g. a file collector), or a composite collector (e.g. a file collector as well as an output stream collector). In the case of the composite collector, the entry is added to all its collectors. The user must define their collectors in their code and register them with the \textit{DataCollectionManager}.

\subsection{Extending the DataCollection Library}
If the user wishes to extend our library with a new type of event, they can do so by implementing a new type of \textit{Event} (or \textit{EventSequence}) and \textit{EventHandler}.

\section{Sample application}


\section{Notes}
\begin{itemize}
\item Weaving of our concern fails.
\end{itemize}

%\begin{center}
%\includegraphics[trim={0cm 0cm 0cm 0cm},clip,scale=0.56]{Ex3.pdf}
%\end{center}

%----------------------------------------------------------------------------------------

\end{document}